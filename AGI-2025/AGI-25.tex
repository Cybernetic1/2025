\documentclass[runningheads]{llncs}

\usepackage[backend=biber]{biblatex}
\bibliography{AGI-book}

\usepackage[T1]{fontenc}
% T1 fonts will be used to generate the final print and online PDFs,
% so please use T1 fonts in your manuscript whenever possible.
% Other font encondings may result in incorrect characters.
%
\usepackage{graphicx}
% Used for displaying a sample figure. If possible, figure files should
% be included in EPS format.
%
% If you use the hyperref package, please uncomment the following two lines
% to display URLs in blue roman font according to Springer's eBook style:
%\usepackage{color}
%\renewcommand\UrlFont{\color{blue}\rmfamily}

\usepackage{amsmath}
\usepackage{amssymb}    % for \rightsquigarrow
\usepackage{wasysym}	% for frown face
\usepackage{mathrsfs} 	% for \mathscr
\usepackage{stmaryrd}
\usepackage{mathtools}		% to extend length of double harpoon arrows
\usepackage{tikz}
\usetikzlibrary{arrows,positioning,decorations.markings,shapes.multipart,shapes.geometric}
\usepackage{tikz-cd}

%\newcommand{\selfLoop}[2]{\begin{tikzcd}
%	#1\ar[in=30,out=60,loop,"#2"{pos=.7}]
%	\end{tikzcd}}

\makeatletter
\newcommand\mathcircled[1]{%
	\mathpalette\@mathcircled{#1}%
}
\newcommand\@mathcircled[3]{%
	\tikz[baseline=(math.base)] \node[draw,circle,inner sep=1pt] (math) {$\m@th#1#2$};%
}
\makeatother

\begin{document}
%
\title{Combining LLM and RL and Logic Transformer}
%
%\titlerunning{Abbreviated paper title}
% If the paper title is too long for the running head, you can set
% an abbreviated paper title here
%
\author{King-Yin Yan \orcidID{0009-0007-8238-2442}}
%
\authorrunning{K-Y. Yan}
% First names are abbreviated in the running head.
% If there are more than two authors, 'et al.' is used.
%
\institute{\email{general.intelligence@gmail.com}}
%
\maketitle              % typeset the header of the contribution
%
\begin{abstract}
blah

\keywords{AGI \and large language models \and reinforcement learning \and neural-symbolic integration}
\end{abstract}

\section{LLM + RL architectures}

For ``string diagrams'' there are usually two conventions: 1) data are nodes, functions are edges: \tikz[baseline=(math.base)] \node[draw,circle,inner sep=1pt] (math) {$x$}; $\stackrel{f}{\longrightarrow}$ \tikz[baseline=(math.base)] \node[draw,circle,inner sep=1pt] (math) {$y$};  or alternatively 2) functions are nodes, data are edges: $\stackrel{x}{\longrightarrow}$ \tikz[baseline=(math.base)] \node[draw,circle,inner sep=1pt,fill=gray!20] (math) {$f$}; $\stackrel{y}{\longrightarrow}$.  In the following, I make explicit nodes for both functions (grey) and data (white), whereas edges merely represent linkages.

First we look at the \textbf{fundamental forms} of RL (left) and auto-encoders (right).  The eye represents observations and the mouth (speech) actions.  Because RL has to maximize rewards, its internal representation (the state) must eventually approach a good approximation of the world.  The auto-encoder, of which LLMs are a special case, works by compressing world-data (via $\rho$) and de-compressing (via $\rho^{-1}$) to re-construct the data (grey world).
\begin{equation}
\begin{tikzpicture}[every path/.style={ultra thick},,decoration={
	markings,mark=at position 0.53 with {\arrow{>}}}]

\node[] (name) at (1,2) {\textbf{(RL)}};

\node[] (eye) at (-3, 0.7) {\includegraphics{eye-symbol.png}};
\node[] (mouth) at (-3, 0.2) {\includegraphics{mouth-symbol.png}};
\node[draw,ultra thick,rectangle] (state) at (0, 0.5) {state};
\node[draw,ultra thick,circle,fill=gray!20] (RL) at (2, 0.5) {RL};

\node[] (world) at (-1.3,1.8) {\includegraphics[scale=0.6]{world-symbol.png}};
\node[rotate=-40] (approx) at (-0.6,1.2) {\LARGE$\approx$};

\draw[postaction={decorate},rounded corners=12pt] (state.north) to ([yshift=18pt]state.north) to ([yshift=12pt]RL.north) to (RL.north);
\draw[postaction={decorate},rounded corners=12pt] (RL.south) to ([yshift=-12pt]RL.south) to ([yshift=-18pt]state.south) to (state.south);

\draw[-left to,shorten >= 5pt] (eye) -- (state.north west);
\draw[-left to,shorten <= 5pt] (state.south west) -- (mouth);
\end{tikzpicture}
\qquad \quad
\begin{tikzpicture}[node distance=-2pt,every path/.style={ultra thick},every text node part/.style={align=center},decoration={
	markings,mark=at position 0.53 with {\arrow{>}}}]

\node[] (name) at (0,1.5) {\textbf{(Auto-encoder)}};

\node[] (world1) at (-2,0) {\includegraphics[scale=0.6]{world-symbol.png}};
\node[] (world2) at (2,0) {\includegraphics[scale=0.6]{world-gray.png}};

\node[rotate=90,rectangle,fill=white] (state) at (0, 0) {state};
\draw[] (state.south east) -- (state.north east) (state.south west) -- (state.north west);

\node (compress1) [rotate=90, draw,trapezium, trapezium angle=-70, fill=gray!20, minimum height=32pt, inner xsep=3.1pt, above=0pt of state] {};
\node (compress2) [rotate=90, draw,trapezium, trapezium angle=70, fill=gray!20, minimum height=32pt, inner xsep=3.1pt, below=0pt of state] {};

\node[] (rho) at (0.1,0) {\large $\rho \qquad \quad \; \rho^{-1}$};
%\node[] (rho-1) at (1,0) {$\rho^{-1}$};

\end{tikzpicture}
\end{equation}

% 反省的有效性,取决于: 这个 loop 如何训练,有效地训练。

Next we look at two types of architectures for combining RL and LLM (cite).  Type A is the ``mainstream'' approach, of which RLHF seems to be one instance.  Here, ...  \textit{The ``format'' of human internal thought is natural language.}
\begin{equation}
\hspace{-0.5cm}
\begin{tikzpicture}[node distance=-2pt,every path/.style={ultra thick},every text node part/.style={align=center},decoration={
	markings,mark=at position 0.53 with {\arrow{>}}}]

\node[] (name) at (0.1,2) {\textbf{(Type A)}};

\node[draw,ultra thick,circle,fill=gray!20] (RL) at (2, 0) {RL};
\node (state) [draw,rectangle split,rectangle split parts=2,minimum width=5em] at (0, 0) {out-state \nodepart{two} in-state};

\node[] (speech1) at (-3,2) {\includegraphics[scale=0.6]{speech-gray.png}};
\node[] (speech2) at (-3,-2) {\includegraphics[scale=0.6]{speech-symbol.png}};

\node (compress1) [draw,trapezium,trapezium angle=-70, fill=gray!20, minimum height=32pt, inner xsep=-2pt] at (-3,0.55) {NL};
\node (compress2) [draw,trapezium,trapezium angle=70, fill=gray!20, minimum height=32pt, inner xsep=-7pt, below=of compress1] {NL${}^{-1}$};

\node (stateA) [rectangle,minimum width=45pt, minimum height=8pt, above=0pt of compress1] {};
\draw(stateA.south west) -- (stateA.north west) -- (stateA.north east) -- (stateA.south east);
\node (stateB) [rectangle,minimum width=45pt, minimum height=8pt, below=0pt of compress2] {};
\draw(stateB.north west) -- (stateB.south west) -- (stateB.south east) -- (stateB.north east);

\draw[dashed] (stateA.east) to ([yshift=-5pt]state.north west);
\draw[dashed] (stateB.east) to ([yshift=5pt]state.south west);

\draw[postaction={decorate},rounded corners=12pt] (state.north) to ([yshift=14pt]state.north) to ([yshift=15pt]RL.north) to (RL.north);
\draw[postaction={decorate},rounded corners=12pt] (RL.south) to ([yshift=-15pt]RL.south) to ([yshift=-14pt]state.south) to (state.south);
\end{tikzpicture}
\qquad \qquad
\begin{tikzpicture}[node distance=-2pt,every path/.style={ultra thick},every text node part/.style={align=center},decoration={
	markings,mark=at position 0.53 with {\arrow{>}}}]

\node[] (name) at (0.1,2) {\textbf{(Type B)}};

\node[draw,ultra thick,circle,fill=gray!20] (RL) at (2, 0) {RL};
\node (state) [draw,rectangle split,rectangle split parts=2,minimum width=5em] at (0, 0) {out-state \nodepart{two} in-state};

\node[] (world1) at (-3,2) {\includegraphics[scale=0.6]{world-gray.png}};
\node[] (world2) at (-3,-2) {\includegraphics[scale=0.6]{world-symbol.png}};

\node (compress1) [draw,trapezium,trapezium angle=-70, fill=gray!20, minimum height=40pt, inner xsep=2pt] at (-3,0.67) {};
\node (compress2) [draw,trapezium,trapezium angle=70, fill=gray!20, minimum height=40pt, inner xsep=2pt, below=of compress1] {};

\node (stateA) [draw,trapezium,trapezium angle=-70, fill=white, minimum height=10pt, inner xsep=12pt, above=-25pt of compress1] {};
%\draw(stateA.south west) -- (stateA.north west) -- (stateA.north east) -- (stateA.south east);
\node (stateB) [draw,trapezium,trapezium angle=70, fill=white, minimum height=10pt, inner xsep=12pt, below=-25pt of compress2] {};
%\draw(stateB.north west) -- (stateB.south west) -- (stateB.south east) -- (stateB.north east);

\draw[dashed] (stateA.east) to ([yshift=-5pt]state.north west);
\draw[dashed] (stateB.east) to ([yshift=5pt]state.south west);

\draw[postaction={decorate},rounded corners=12pt] (state.north) to ([yshift=14pt]state.north) to ([yshift=15pt]RL.north) to (RL.north);
\draw[postaction={decorate},rounded corners=12pt] (RL.south) to ([yshift=-15pt]RL.south) to ([yshift=-14pt]state.south) to (state.south);
\end{tikzpicture}
\end{equation}

%
% ---- Bibliography ----
%
% BibTeX users should specify bibliography style 'splncs04'.
% References will then be sorted and formatted in the correct style.
%
% \bibliographystyle{splncs04}
\printbibliography

\end{document}
