\input{../YKY-preamble.tex}
% \usepackage[no-math]{fontspec}
% \setmainfont[BoldFont=Alibaba_Sans_Regular.otf,ItalicFont=Alibaba_Sans_Light_Italic.otf]{Alibaba_Sans_Light.otf}

\usepackage[backend=biber]{biblatex}
\bibliography{../AGI-book}

\usepackage[active,tightpage]{preview}		% for continuous page(s)
\renewcommand{\PreviewBorder}{0.5cm}
\renewcommand{\thempfootnote}{\arabic{mpfootnote}}

% \usepackage[absolute,overlay]{textpos}		% for page number on upper left corner

\usepackage{color}
% \usepackage{mathtools}
\usepackage[hyperfootnotes=false]{hyperref}

% \usepackage[backend=biber,style=numeric]{biblatex}
% \bibliography{../AGI-book}
% \renewcommand*{\bibfont}{\footnotesize}

\usetikzlibrary{shapes}
% \usepackage[export]{adjustbox}	% ??
\usepackage{verbatim} % for comments
% \usepackage{newtxtext,newtxmath}	% Times New Roman font

% \titleformat{\subsection}[hang]{\bfseries\large\color{blue}}{}{0pt}{}
% \numberwithin{equation}{subsection}

\newcommand{\underdash}[1]{%
	\tikz[baseline=(toUnderline.base)]{
		\node[inner sep=1pt,outer sep=10pt] (toUnderline) {#1};
		\draw[dashed] ([yshift=-0pt]toUnderline.south west) -- ([yshift=-0pt]toUnderline.south east);
	}%
}%

\newcommand\reduline{\bgroup\markoverwith{\textcolor{red}{\rule[-0.5ex]{2pt}{0.4pt}}}\ULon}

%\DeclareSymbolFont{symbolsC}{U}{txsyc}{m}{n}
%\DeclareMathSymbol{\strictif}{\mathrel}{symbolsC}{74}
%\DeclareSymbolFont{AMSb}{U}{msb}{m}{n}
%\DeclareSymbolFontAlphabet{\mathbb}{AMSb}
%\setmathfont{lmroman17-regular.otf}
\DeclareMathOperator*{\argmin}{arg\,min}
\DeclareMathOperator*{\argmax}{arg\,max}

% \usepackage[most]{tcolorbox}
%\tcbset{on line,
%	boxsep=4pt, left=0pt,right=0pt,top=0pt,bottom=0pt,
%	colframe=red,colback=pink,
%	highlight math style={enhanced}
%}
%\newcommand{\atom}{\vcenter{\hbox{\tcbox{....}}}}

\let\oldtextbf\textbf
\renewcommand{\textbf}[1]{\textcolor{blue}{\oldtextbf{#1}}}

\newcommand{\logic}[1]{{\color{violet}{\textit{#1}}}}
\newcommand{\underconst}{\includegraphics[scale=0.5]{../2020/UnderConst.png}}
\newcommand{\KBsymbol}{\vcenter{\hbox{\includegraphics[scale=1]{../KB-symbol.png}}}}
\newcommand{\bbOmega}{\vcenter{\hbox{\includegraphics[scale=1]{../bbOmega-symbol.png}}}}
\newcommand{\token}{\vcenter{\hbox{\includegraphics[scale=1]{token.png}}}}
\newcommand{\proposition}{\vcenter{\hbox{\includegraphics[scale=0.8]{proposition.png}}}}

\begin{document}

\begin{preview}

% \noindent {\color{cyan}``\textit{If men cease to believe that they will one day become gods then they will surely become worms.}''
% \hfill --- Henry Miller}

\title{\vspace{-0.5cm} \bfseries\color{blue}{\Huge \cc{逻辑 与 auto-encoder}{Logic and auto-encoders}}}

\author{\small YKY [\today]} % Your name
\date{\vspace{-0.5cm}} % Date, can be changed to a custom date
% \date{\small \today}

\maketitle

\setcounter{section}{-1}
\newcounter{mypage}
\setcounter{mypage}{0}

\begin{minipage}{\textwidth}
\setlength{\parskip}{0.4\baselineskip}

现在问题是: 逻辑与 AE 的关系为何? 逻辑公理集 是 世界的 generator.  世界是可以用 AE 学习的。

在 TicTacToe 问题里,世界有某种演化的方式,但我们也可以下 actions。

Next thought operator 和 reconstruct world 之间的关系...  之前好像已经解决了? 但在 TicTacToe 环境下似乎要再厘清一次。  逻辑公理的作用是 model 世界? 但另一种做法是推导最佳 actions.  这个问题似乎跟 LLM = world model or language model 是一样的? 是用逻辑推导世界还是推导 actions?  似乎两者都可以。 AE 可以是世界模型,也可以是 next thought 模型。 我有兴趣的是 next thought 模型。 但其实有趣的是用 AE 建立世界模型,然后再在其中玩 RL.  暂时的目的是让它产生一套能玩 RL 的语言。

我们的目的是用 TicTacToe 模拟 AGI 的训练。 AGI 首先透过预测世界,建立世界模型。 问题是之后的 RL 是怎样定义的? How to define states and rewards?  明显的答案就是,state = 命题集合,reward 是由某些命题的存在而 trigger 的。

另一个问题是: AE 的 latent state 是什么? 它是世界当前状态的压缩? 如此说来,则它跟 RL 状态是一样的。

逻辑 rule 如何表示 随机策略?

\end{minipage}
\end{preview}

\end{document}
