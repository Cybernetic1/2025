\input{../YKY-preamble.tex}
% \usepackage[no-math]{fontspec}
% \setmainfont[BoldFont=Alibaba_Sans_Regular.otf,ItalicFont=Alibaba_Sans_Light_Italic.otf]{Alibaba_Sans_Light.otf}

\usepackage[backend=biber]{biblatex}
\bibliography{../AGI-book}

\usepackage[active,tightpage]{preview}		% for continuous page(s)
\renewcommand{\PreviewBorder}{0.5cm}
\renewcommand{\thempfootnote}{\arabic{mpfootnote}}

% \usepackage[absolute,overlay]{textpos}		% for page number on upper left corner

\usepackage{color}
% \usepackage{mathtools}
\usepackage[hyperfootnotes=false]{hyperref}

% \usepackage[backend=biber,style=numeric]{biblatex}
% \bibliography{../AGI-book}
% \renewcommand*{\bibfont}{\footnotesize}

\usetikzlibrary{shapes}
% \usepackage[export]{adjustbox}	% ??
\usepackage{verbatim} % for comments
% \usepackage{newtxtext,newtxmath}	% Times New Roman font

% \titleformat{\subsection}[hang]{\bfseries\large\color{blue}}{}{0pt}{}
% \numberwithin{equation}{subsection}

\newcommand{\underdash}[1]{%
	\tikz[baseline=(toUnderline.base)]{
		\node[inner sep=1pt,outer sep=10pt] (toUnderline) {#1};
		\draw[dashed] ([yshift=-0pt]toUnderline.south west) -- ([yshift=-0pt]toUnderline.south east);
	}%
}%

\newcommand\reduline{\bgroup\markoverwith{\textcolor{red}{\rule[-0.5ex]{2pt}{0.4pt}}}\ULon}

%\DeclareSymbolFont{symbolsC}{U}{txsyc}{m}{n}
%\DeclareMathSymbol{\strictif}{\mathrel}{symbolsC}{74}
%\DeclareSymbolFont{AMSb}{U}{msb}{m}{n}
%\DeclareSymbolFontAlphabet{\mathbb}{AMSb}
%\setmathfont{lmroman17-regular.otf}
\DeclareMathOperator*{\argmin}{arg\,min}
\DeclareMathOperator*{\argmax}{arg\,max}

% \usepackage[most]{tcolorbox}
%\tcbset{on line,
%	boxsep=4pt, left=0pt,right=0pt,top=0pt,bottom=0pt,
%	colframe=red,colback=pink,
%	highlight math style={enhanced}
%}
%\newcommand{\atom}{\vcenter{\hbox{\tcbox{....}}}}

\let\oldtextbf\textbf
\renewcommand{\textbf}[1]{\textcolor{blue}{\oldtextbf{#1}}}

\newcommand{\logic}[1]{{\color{violet}{\textit{#1}}}}
\newcommand{\underconst}{\includegraphics[scale=0.5]{../2020/UnderConst.png}}
\newcommand{\KBsymbol}{\vcenter{\hbox{\includegraphics[scale=1]{../KB-symbol.png}}}}
\newcommand{\token}{\vcenter{\hbox{\includegraphics[scale=1]{token.png}}}}
\newcommand{\proposition}{\vcenter{\hbox{\includegraphics[scale=0.8]{proposition.png}}}}

\begin{document}

\begin{preview}

% \noindent {\color{cyan}``\textit{If men cease to believe that they will one day become gods then they will surely become worms.}''
% \hfill --- Henry Miller}

\title{\vspace{-0.5cm} \bfseries\color{blue}{\Huge \cc{逻辑公式的「空间形状」}{The ``Shape'' of Logic Rules}}}

\author{\small YKY [\today]} % Your name
\date{\vspace{-0.5cm}} % Date, can be changed to a custom date
% \date{\small \today}

\maketitle

\setcounter{section}{-1}
\newcounter{mypage}
\setcounter{mypage}{0}

% (1) Circled page number on upper left corner
%\begin{textblock*}{5cm}(2.1cm,2.3cm) % {block width} (coords)
%{\color{red}{\large \textcircled{\small \themypage}}}
%\addtocounter{mypage}{1}
%\end{textblock*}

\begin{minipage}{\textwidth}
\setlength{\parskip}{0.4\baselineskip}

看待 AGI 的其中一种观点、当然也是我最喜欢的观点,就是说 AGI 的目的,是 \textbf{学习一套逻辑公式去描述世界}。

这套逻辑公式本来是不存在的,它是从机器学习的过程中「无中生有」的。 但既然逻辑法则可以任由我们创造,而目的是 maximize rewards,则似乎这个问题 ``under-specified,'' 也就是说约束条件太弱。 其实它的约束条件是因为 \textbf{记忆有限},換句话说是一个 \textbf{资讯压缩} 的问题。 所以这个问题是 well-defined 而且有 solution, 在数学上是一个很有意思的问题。 本文试图准确地 描述 逻辑公式的空间结构。

首先,如果大家熟悉微分方程的,应该见过所谓 vector field 里的 ``flow''(向量场的流动):
\begin{equation}
	\vcenter{\hbox{\includegraphics[scale=0.6]{vector-field-flow.png}}}
\end{equation}
在状态空间里,我们会走出一条 \textbf{轨迹} (trajectory), 在轨迹上会收到 \textbf{奖励} (rewards).  强化学习的目的就是 maximize 在长远的 time horizon 上的奖励总和。

注意 以上的图像是 \textbf{连续的},但 AGI 的符号逻辑的状态是 \textbf{离散的}。 离散的 强化学习 服从 Bellman 方程,而 连续的 control theory 服从 Hamilton-Jacobi 方程,它描述一个粒子在某力场之下的运动方式,后者的向量流称为 Hamiltonian flow.  有时我会在这两个图像之间跳了跳去,以获得某些 insights,但这也不是必需的,只是我也稍为熟悉物理那边,所以比较方便。

在 AGI 里,状态 = Working Memory 的内涵,每个状态就是一个「故事」。 比如说,$x_0 =$「现在是凌晨  3am $\wedge$ 我很肚饿 $\wedge$ 冰箱又没有食物 $\wedge$ 钱包也没现金。」 或者 $x_7 =$「我很爱她 $\wedge$ 但她不爱我 $\wedge$ 昨天还被她扇了一巴掌。」

而 vector field 则代表每个状态可以如何 transition 到另一状态。 换句话说,vector field \textbf{等价于} 我们的逻辑知识库,但逻辑以特殊的方式定义每个状态点上的 tangent vector;  通常一个逻辑公式可以定义很多个状态上的 tangent vectors.  因此这个 向量场具有特殊的逻辑结构,形成数学上有趣的问题。

\centering $\mathbb{X} =$ \oldtextbf{state space (状态空间)} \vspace{-1em}
\begin{equation}
	\vcenter{\hbox{\includegraphics[scale=0.7]{state-space.png}}}
	\qquad
	\begin{tabular}{c}
		顺着向量场的状态转移\\
		$ {\color{red} \bullet} \rightarrow {\color{green} \bullet} $ \\
		$ x_0 \mapsto x_1 $ \\
	\end{tabular}
\end{equation}

\centering $\mathbb{X} =$ \oldtextbf{proposition space (命题空间)} \vspace{-1em}
\begin{equation}
	\vcenter{\hbox{\includegraphics[scale=0.7]{proposition-space.png}}}
	\qquad
	\begin{tabular}{l}
	$ {\color{red} \bullet} = \{ \boxed{1}, \boxed{2}, {\color{red} \boxed{3}, \boxed{4} } \} $ \\
	$ {\color{green} \bullet} = \{ \boxed{1}, \boxed{2}, {\color{green} \boxed{5}} \} $ \\
	$ {\color{red} \boxed{3}} \wedge {\color{red} \boxed{4}} \rightarrow {\color{green}\boxed{5}} $ \\
\end{tabular}
\end{equation}

\centering $\mathbb{X} =$ \oldtextbf{symbol space (符号空间)} \vspace{-1em}
\begin{equation}
	\vcenter{\hbox{\includegraphics[scale=0.7]{object-space.png}}}
	\qquad
	\begin{tabular}{l}
	$ {\color{red} \boxed{3}} = \mathrm{father}(a,b) $ \\
	$ {\color{red} \boxed{4}} = \mathrm{father}(b,c) $ \\
	$ {\color{green} \boxed{5}} = \mathrm{grand\text{-}father}(a,c) $ \\
	\end{tabular}
\end{equation}

\end{minipage}
\end{preview}

\end{document}
